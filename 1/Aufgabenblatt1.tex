\documentclass{article}

\usepackage[utf8]{inputenc}
\usepackage[ngerman]{babel}

\usepackage{amssymb}
\usepackage{amsmath}

\usepackage{latexsym}

\title{FGI2 Übungen Blatt 1}

\author{Oliver Sengpiel, 6322763 \\
	Daniel Speck, 63XXXXX \\
	Daniel Krempels, 6YYYYYY}

\begin{document}

\maketitle

\section{1.3}
\subsection{1.3.1}
$L(A_n)$ als regulärer Ausdruck: \\
$L(A_n) = (a^{2i}\cdot c\cdot b^{2i}) + (a^{2i-1}\cdot d\cdot b^{2i-1}) +
(a^{n}\cdot d) + (d)$ mit $i \in \{1, \dots, \frac{n}{2} \}$

\subsection{1.3.3}
%dachte ich fang mal an mit den nervigen plain-text sachen
Sei $M(A_n)$ genau die vom Automaten akzeptierte Sprache. \\
$L(A_n) \subseteq M(A_n)$: 

$L(A_n)$ wird vom Automaten akzeptiert. Wird vom Startzustand aus ein einziges 'd' gelesen, so geht der Automat direkt in den Endzustand $p_1$ über und akzeptiert. Wird eine gerade Anzahl an 'a's gelesen, so erreicht man einen Zustand $p_i$ mit i mod 4 = 0. Von hier aus kann das folgende 'd' gelesen werden, sowie dieselbe Anzahl an 'b's wie 'a's, hiermit wird auch der Endzustand $p_1$ erreicht. Wird eine ungerade Anzahl an 'a's gelesen, so wird ein Zustand $p_i$ mit i mod 4 = 2 erreicht. Von diesen aus kann das darauf folgende 'c' gelesen werden und wiederum die selbe Anzahl an 'b's wie 'a's, und derselbe Endzustand $p_1$ wird erreicht. Somit werden alle Eingaben von $L(A_n)$ akzeptiert. \\
$M(A_n) \subseteq L(A_n)$: 

Alle vom Automaten akzeptierten Wörter sind in $L(A_n)$ enthalten. Sei $w \in M(A_n)$. $w$ kann vier verschiedene Formen haben: 1. $w$ kann aus einem einzelnen 'd' bestehen, dies ist in $L(A_n)$ enthalten. 2. $w$ kann auch aus einer beliebigen, geraden Anzahl an 'a's, darauf folgend ein 'd' und darauf folgend genau so viele 'b's wie 'a's bestehen. Auch in diesem Fall gilt $w \in L(A_n)$. 3. Oder $w$ ist aus einer ungeraden Anzahl an 'a's und der gleichen Anzahl an 'b's aufgebaut, genau zwischen 'a's und b's ein 'c'. Auch dieses $w$ ist in $L(A_n)$ enthalten. Somit sind alle Wörter, die vom Automaten gelesen werden können, auch in $L(A_n)$, also gilt: $M(A_n) \subset L(A_n)$. 

\subsection{1.3.4}

$L(A_n)$ ist regulär. Denn in der akzeptierten Sprache ist festgelegt, dass kein Wort länger als $n + 1, n$ Anzahl der Zustände sein kann. Damit kann es keine Schleifen in dem Wort geben und das Pumping Lemma nicht widerlegt werden.

 \subsection{1.3.5}
 
 $A = \bigcup_{n \geq 0} L(A_n)$ . Nun ist gegeben, dass n beliebig ist. Damit A regulär ist, und also von einem endlichen Automaten akzeptiert wird, muss es eine Schleife in einem Wort $w \in (a^{2i}\cdot c\cdot b^{2i})$ geben, bei dem $i \g n, n$ Anzahl der Zustände des Automaten. 
 

\end{document}

% d acb aadbb aaacbbb ... aaaaaaaaaa//aaaaad
% a^2i d b^2i + a^2i+1 c b^2i+1 + a^n d

